% *****************************************************************
% Estout related things
% http://www.jwe.cc/2012/03/stata-latex-tables-estout/
% *****************************************************************

%\newcommand{\sym}[1]{\rlap{#1}}% Thanks to David Carlisle
\def\sym#1{\ifmmode^{#1}\else\(^{#1}\)\fi}

\let\estinput=\input% define a new input command so that we can still flatten the document

% classic functions
% estwide and estalt
% don't let you specify the model titles, panel title, etc

\newcommand{\estwideclassic}[3]{
  \vspace{.75ex}{
    \begin{tabular*}
      {\textwidth}{@{\hskip\tabcolsep\extracolsep\fill}l*{#2}{#3}}
      \toprule
      \estinput{#1}
      \bottomrule
      \addlinespace[.75ex]
    \end{tabular*}
  }
}

\newcommand{\estautoclassic}[3]{
  \vspace{.75ex}{
    \begin{tabular}{l*{#2}{#3}}
      \toprule
      \estinput{#1}
      \bottomrule
      \addlinespace[.75ex]
    \end{tabular}
  }
}

% augmented functions allow you to specify the model titles
% but are not designed for multiple panels (tables stacked on top of eachother)

\newcommand{\estauto}[4]{
  \vspace{.75ex}{
    \begin{tabular}{l*{#2}{#3}}
      \toprule
      #4
      \estinput{#1}
      \bottomrule
      \addlinespace[.75ex]
    \end{tabular}
  }
}

\newcommand{\estwide}[4]{
  \vspace{.75ex}{
    \begin{tabular*}
      {\textwidth}{@{\hskip\tabcolsep\extracolsep\fill}l*{#2}{#3}}
      \toprule
      #4
      \estinput{#1}
      \bottomrule
      \addlinespace[1.75ex]
    \end{tabular*}
  }
}

% panel functions

\newcommand{\estautopanel}[6]{
  \vspace{.75ex}{
    \begin{tabular}{l*{#2}{#3}}
      \toprule
      &\multicolumn{#2}{c}{#4}\\\cmidrule(lr){#5}
      #6
      \estinput{#1}
      \bottomrule
      \addlinespace[.75ex]
    \end{tabular}
  }
}

\newcommand{\estwidepanel}[6]{
  \vspace{.75ex}{
    \begin{tabular*}
      {\textwidth}{@{\hskip\tabcolsep\extracolsep\fill}l*{#2}{#3}}
      \toprule
      &\multicolumn{#2}{c}{#4}\\\cmidrule(lr){#5}
      #6
      \estinput{#1}
      \bottomrule
      \addlinespace[1.75ex]
    \end{tabular*}
  }
}

% panel top middle and bottoms

\newcommand{\estautopaneltop}[6]{
  \vspace{.75ex}{
    \begin{tabular}{l*{#2}{#3}}
      \toprule
      &\multicolumn{#2}{c}{#4}\\\cmidrule(lr){#5}
      #6
      \estinput{#1}
      \addlinespace[.75ex]
    \end{tabular}
  }
}

\newcommand{\estautopanelmiddle}[6]{
  \vspace{.75ex}{
    \begin{tabular}{l*{#2}{#3}}
      &\multicolumn{#2}{c}{#4}\\\cmidrule(lr){#5}
      #6
      \estinput{#1}
      \addlinespace[.75ex]
    \end{tabular}
  }
}

\newcommand{\estautopanelbottom}[6]{
  \vspace{.75ex}{
    \begin{tabular}{l*{#2}{#3}}
      &\multicolumn{#2}{c}{#4}\\\cmidrule(lr){#5}
      #6
      \estinput{#1}
      \bottomrule
      \addlinespace[.75ex]
    \end{tabular}
  }
}

\newcommand{\estwidepaneltop}[6]{
  \vspace{.75ex}{
    \begin{tabular*}
      {\textwidth}{@{\hskip\tabcolsep\extracolsep\fill}l*{#2}{#3}}
      \toprule
      &\multicolumn{#2}{c}{#4}\\\cmidrule(lr){#5}
      #6
      \estinput{#1}
      \addlinespace[1.75ex]
    \end{tabular*}
  }
}

\newcommand{\estwidepanelmiddle}[6]{
  \vspace{.75ex}{
    \begin{tabular*}
      {\textwidth}{@{\hskip\tabcolsep\extracolsep\fill}l*{#2}{#3}}
      &\multicolumn{#2}{c}{#4}\\\cmidrule(lr){#5}
      #6
      \estinput{#1}
      \addlinespace[.75ex]
    \end{tabular*}
  }
}

\newcommand{\estwidepanelbottom}[6]{
  \vspace{.75ex}{
    \begin{tabular*}
      {\textwidth}{@{\hskip\tabcolsep\extracolsep\fill}l*{#2}{#3}}
      &\multicolumn{#2}{c}{#4}\\\cmidrule(lr){#5}
      #6
      \estinput{#1}
      \bottomrule
      \addlinespace[.75ex]
    \end{tabular*}
  }
}

% beamer tables
% should never be stacked panels

\newcommand{\estautobeamer}[4]{
  \vspace{.75ex}{
    \begin{tabular}{l*{#2}{#3}}
      #4
      \estinput{#1}
      \addlinespace[.75ex]
    \end{tabular}
  }
}

\newcommand{\estwidebeamer}[4]{
  \vspace{.75ex}{
    \begin{tabular*}
      {\textwidth}{@{\hskip\tabcolsep\extracolsep\fill}l*{#2}{#3}}
      #4
      \estinput{#1}
      \addlinespace[1.75ex]
    \end{tabular*}
  }
}

% Allow line breaks with \\ in specialcells
\newcommand{\specialcell}[2][c]{%
\begin{tabular}[#1]{@{}c@{}}#2\end{tabular}}

% *****************************************************************
% Custom subcaptions
% *****************************************************************

\usepackage{threeparttable}

% Note/Source/Text after Tables
\newcommand{\Figtext}[1]{%
  \begin{tablenotes}[para,flushleft]
  \hspace{6pt}
  \hangindent=1.75em
  #1
  \end{tablenotes}
}

\newcommand{\Fignote}[1]{\Figtext{\emph{Note:~}~#1}}
\newcommand{\Figsource}[1]{\Figtext{\emph{Source:~}~#1}}
\newcommand{\Figsources}[1]{\Figtext{\emph{Sources:~}~#1}}
\newcommand{\Starnote}{\Figtext{* $p < 0.1$, ** $p < 0.05$, *** $p < 0.01$}}% Add significance note with \starnote


% *****************************************************************
% siunitx helps line things up in tables
% *****************************************************************

% centering in tables
\usepackage{siunitx}
\sisetup{
  %detect-mode,
  tight-spacing		= true,
  group-digits		= false ,
  input-signs		= ,
  input-symbols		= ( ) [ ] - + *,
  input-open-uncertainty	= ,
  input-close-uncertainty	= ,
  table-align-text-post	= false
}

% *****************************************************************
% Wider command to fit big tables
% *****************************************************************

\newcommand\Wider[2][3em]{%
  \makebox[\linewidth][c]{%
    \begin{minipage}{\dimexpr\textwidth+#1\relax}
    \raggedright#2
    \end{minipage}%
  }%
}
